\documentclass[aip,jcp,reprint,floatfix,a4paper]{revtex4-1}  %,linenumbers
\usepackage{cmap}
\usepackage[T1]{fontenc}
\usepackage[Euler]{upgreek}
\usepackage{times}
\usepackage{helvet}
\usepackage{fixmath}
\usepackage{graphicx}
\usepackage{amsmath}
\usepackage{geometry}
\geometry{verbose,a4paper,tmargin=2.5cm,bmargin=3cm,lmargin=1.5cm,rmargin=1.5cm,headheight=13.6pt}
\def\pgfsysdriver{pgfsys-dvips.def}
\usepackage[dvipsnames]{xcolor}
\usepackage{tikz,pgfplots}
\usetikzlibrary{backgrounds, calc, arrows}
\usetikzlibrary{positioning}
\tikzstyle{every picture}+=[remember picture]
\pgfplotsset{compat=newest}
\usepgfplotslibrary{external}
\tikzexternalize[up to date check=md5]{document}
\tikzsetexternalprefix{figures/}
\tikzset{external/system call={latex \tikzexternalcheckshellescape
-halt-on-error -interaction=batchmode -jobname "\image" "\texsource"; dvips -E
-o "\image".eps "\image".dvi; epstool --copy --bbox "\image".eps
"\image"new.eps; mv "\image"new.eps "\image".eps ;
 find ${PWD}/figures -maxdepth 1 -type f ! -iname
 "*.eps" -delete; }}

\usepackage[nodayofweek]{datetime}
\newdateformat{myDate}{\THEDAY\ \monthname[\THEMONTH] \THEYEAR}

\usepackage{bbm}


\begin{document}

\title[]{Implementation of the bilayer membrane model proposed by Cooke \textit{et al.} in LAMMPS}

\author{P. Wirnsberger}
\affiliation{
Department of Chemistry, University of Cambridge.
}

\date{\today}
\maketitle


\section{\label{sec:vattr}Attractive tail-tail interaction}
Tail beads interact according to the radially symmetric, pairwise potential
\begin{equation}
V_\text{attr}(r) = -\epsilon  
  \begin{cases}   
     1                                                                   &  \mbox{if $ r < r_\text{c}$,} \\
     \cos^2\left(\frac{\pi (r-r_\text{c})}{2 w_\text{c}} \right)           &  \mbox{if $ r_\text{c} \leq r \leq r_\text{c} + w_\text{c}$,} \\
     0                                                                   &  \mbox {otherwise},
  \end{cases}
\end{equation}
where $r$ is the distance between two tails, $\epsilon$ defines the unit of energy, 
$w_\text{c}$ defines the decay range and $r_\text{c}$ is the cutoff for the repulsive potential $V_\text{rep}$.
The force exerted on particle $i$, located at $\mathbold r_i$, by particle $j$, located at $\mathbold r_j$, is given by 
\begin{align}
\mathbold f_{ij} &= -\nabla_{{\mathbold r}_i} V_\text{attr}(r_{ij}) \\
                 &= - V'_\text{attr}(r) \bigg|_{r = r_{ij}} \frac{\mathbold r_{ij}}{r_{ij}}  \\
                 &= -\frac{\pi \epsilon}{2 w_\text{c}}
  \begin{cases}   
     0                                                                   &  \mbox{if $ r < r_\text{c}$,} \\
     \sin\left(\frac{\pi (r-r_\text{c})}{w_\text{c}} \right)             &  \mbox{if $ r_\text{c} \leq r \leq r_\text{c} + w_\text{c}$,} \\
     0                                                                   &  \mbox {otherwise},
  \end{cases}
\end{align}
where $\mathbold r_{ij} = \mathbold r_i - \mathbold r_j$.
\begin{figure*}[b!]
  \centering
  \tikzsetnextfilename{vattr}
  \begin{tikzpicture}[]



\coordinate (shiftV) at  ($(0,0cm)  $);
\begin{scope}[shift=(shiftV), local bounding box=scopeV]

\begin{axis}[%
   xlabel =$r$,
   ylabel= $V_\text{attr}(r)$,
   xmin=0, xmax=6,
   scale only axis, width=7cm , height=4.6cm,%
   legend style={font=\footnotesize,at={(0.9,0.2)},anchor=south 
   east,draw=black,name=legend}]
    

  \addplot[black!35, dotted, forget plot]{ 0.0};
   
  \addplot[solid, blue]   table [x expr=\thisrowno{1}*1, y expr=\thisrowno{2}*1]  {data/v11.dat};
  \addlegendentry{$1-1$}  

  \addplot[dashed, red]   table [x expr=\thisrowno{1}*1, y expr=\thisrowno{2}*1]  {data/v12.dat};
  \addlegendentry{$1-2$}  
 
  \addplot[dotted, thick, purple] table [x expr=\thisrowno{1}*1, y expr=\thisrowno{2}*1]  {data/v22.dat};
  \addlegendentry{$2-2$}  
 
\end{axis}
\end{scope}


\coordinate (shiftF) at  ($1*(9 cm ,0 cm)  $);
\begin{scope}[shift=(shiftF), local bounding box=scopeF]

\begin{axis}[%
   xlabel =$r$,
   ylabel= $-V'_\text{attr}(r)$,
   xmin=0, xmax=6,
   scale only axis, width=7cm , height=4.6cm,%
   legend style={font=\footnotesize,at={(0.9,0.2)},anchor=south 
   east,draw=black,name=legend}]

  \addplot[black!35, dotted, forget plot]{ 0.0};
   
  \addplot[solid, blue]   table [x expr=\thisrowno{1}*1, y expr=\thisrowno{3}*1]  {data/v11.dat};
  \addlegendentry{$1-1$}  

  \addplot[dashed, red]   table [x expr=\thisrowno{1}*1, y expr=\thisrowno{3}*1]  {data/v12.dat};
  \addlegendentry{$1-2$}  
 
  \addplot[dotted, thick, purple] table [x expr=\thisrowno{1}*1, y expr=\thisrowno{3}*1]  {data/v22.dat};
  \addlegendentry{$2-2$}  
 
\end{axis}
\end{scope}


\end{tikzpicture}
 

  \caption{Attractive tail-tail interaction energies \textit{(left)} and forces \textit{(right)}, as computed with LAMMPS ({\scriptsize pair\_write}), for two different tail types
           and the following parameters:
  $(\epsilon_{11}, \epsilon_{12}, \epsilon_{22}) = (1.5,2.0,2.5)$,
  $(w_{\text{c},11}, w_{\text{c},12}, w_{\text{c},22}) = (3.0,3.0,1.0)$ and
  $(r_{\text{c},11}, r_{\text{c},12}, r_{\text{c},22}) = (2.5,2.0,1.5)$.}
  \label{fig:vattr}
\end{figure*}

\end{document}
